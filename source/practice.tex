\section{Практическая часть}

С помощью вольтметра измеряем ЭДС батарейки $\mathcal{E}_x$. Полученное значение:
\[
\mathcal{E}_x = (1{,}461 \pm 0{,}262)\ \text{В}
\]

Собираем схему согласно \cref{fig:3}. Подключаем эталонный источник ЭДС $\mathcal{E}_N = (1{,}018 \pm 0{,}005)\ \text{В}$, выставляем ЭДС источника питания $\mathcal{E} = 3\ \text{В}$, устанавливаем сопротивление $R_3$ на максимальное значение и замыкаем ключ.

Методом подбора сопротивлений $R_1$, $R_2$ и постепенного уменьшения $R_3$ находим значение $R_1$, при котором нуль-гальванометр показывает $+1\ \text{мкА}$, фиксируем $R_1^+$, затем находим $R_1^-$ для $-1\ \text{мкА}$. Среднее значение вычисляем по формуле:
\[
R_{1N} = \frac{R_1^+ + R_1^-}{2}
\]

Аналогичные измерения проводим для ЭДС источника $\mathcal{E} = 6\ \text{В}$ и для измеряемой батарейки $\mathcal{E}_x$. Все полученные данные заносим в \cref{tab:1a,tab:1b}.

\begin{table}[h!]
	\centering
	\caption{Результаты измерений компенсирующих сопротивлений}
	\label{tab:1}
	
	\begin{minipage}{0.45\textwidth}
		\centering
		\subcaption{Экспериментальные данные}
		\label{tab:1a}
		\begin{tabular}{ccc}
			\toprule
			$\mathcal{E}$, В & $3$ & $6$ \\
			\midrule
			$R_{1N_{\text{ср}}}$, Ом & 3\,781,0 & 1\,880,0 \\
			$R_{1x_{\text{ср}}}$, Ом & 5\,425,0 & 2\,707,5 \\
			\bottomrule
		\end{tabular}
	\end{minipage}
	\hfill
	\begin{minipage}{0.45\textwidth}
		\centering
		\subcaption{Теоретические данные}
		\label{tab:1b}
		\begin{tabular}{ccc}
			\toprule
			$\mathcal{E}$, В & $3$ & $6$ \\
			\midrule
			$R_{1N}$, Ом & 3\,770 & 1\,885 \\
			$R_{1x}$, Ом & 5\,411 & 2\,706 \\
			\bottomrule
		\end{tabular}
	\end{minipage}
\end{table}

Также вычислим $R_{1N}$ и $R_{1x}$ по формуле \cref{eq:8} полученная в ре... Далее вычислим $\Epsilon_x$ по формуле \cref{eq:10}, для всех случаев:
\begin{table}[H]
	\centering
	\caption{Теоретические данные}
	\label{tab:2}
	\begin{tabular}{lccc}
		\toprule
		 & $\mathcal{E}$, В & $3$ & $6$ \\
		\midrule
		Метод компенсации & $\Epsilon_{x B}, \text{ В}$ & 1,461 & 1,460 \\
		Вольтметр & $\Epsilon_{x B}, \text{ В}$ & 1,461 & 1,461 \\
		\bottomrule
	\end{tabular}
\end{table}

\subsection{Вычисление погрешностей}

Погрешность $R_{1x}$ и $R_{1N}$ из характеристик приборов:
\[\Delta R_{1x} = \Delta R_{1N} = \pm 11,1 \text{ Ом}\]

Приборная погрешность $\Delta \Epsilon_{x B}$ из характеристик прибора: 
\[\Delta \Epsilon_{x B} = \pm 0,18 \text{ В}\]

\[\Delta \Epsilon_{1N} = \pm 0,005 \text{ В}\]

\[\Delta R_{1x} = \pm \left(0,1 + 0,2\right)\]



Следовательно $\Epsilon_x$ вычисленное методом компенсации имеет вид:
\[\Epsilon_x = (1,460 \pm 0,005) \text{ В}\]

Сравним: Абсолютная погрешность измерения методом компенсации меньше погрешности измерения вольтметром.
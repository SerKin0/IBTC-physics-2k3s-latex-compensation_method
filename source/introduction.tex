\section{Введение}
\subsection*{Цель} \addcontentsline{toc}{subsection}{Цель}
Изучение метода компенсации для измерения электродвижущей силы (ЭДС) источников тока, сравнение его с прямым методом измерения с помощью вольтметра, а также приобретение практических навыков сборки электрических цепей и проведения измерений.

\subsection*{Задачи} \addcontentsline{toc}{subsection}{Задачи}
\begin{enumerate}
	\item Ознакомиться с характеристиками и правилами использования нормального элемента типа НЭ-65. Рассчитать величину защитного сопротивления $R_3$ в схеме \cref{fig:3} в наиболее неблагоприятном случае;
	\item Измерить $\Epsilon_x$ с помощью вольтметра;
	\item Собрать схему, изображенную на \cref{fig:3}, и измерить $\Epsilon_x$ методом компенсации. Перед измерениями целесообразно оценить $R_{1X}$ и $R_{1N}$;
	\item Измерить $\Epsilon_x$ с помощью промышленного компенсатора.
\end{enumerate}

\subsection*{Приборы и оборудование} \addcontentsline{toc}{subsection}{Приборы и оборудование}
\begin{itemize}
	\item Блок питания БП-28;
	\item Нормальный элемент типа НЭ-65 класса 0,005 (эталонная ЭДС $\Epsilon_N \approx 1,0186$ В);
	\item Два одинаковых штепсельных магазина сопротивлениями $R_1$ и $R_2$ ($R_1 + R_2 = R$);
	\item Защитный резистор $R_3$;
	\item Нуль-гальванометр (\verb|НГ|);
	\item Вольтметр;
	\item Источник с неизвестной ЭДС $\Epsilon_x$;
	\item Промышленный компенсатор (готовый прибор для компенсации, например, потенциометр типа ПП-63 или аналогичный).
	\item Провода, переключатели (K1) и другие соединительные элементы.
\end{itemize}
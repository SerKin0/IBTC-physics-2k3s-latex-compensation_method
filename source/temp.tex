\section{Практическая часть}

С помощью вольтметра измеряем ЭДС батарейки $\Epsilon_x$. Полученное значение:
\[
\Epsilon_x = 1,461 \text{В}
\]
Собираем схему согласно \cref{fig:3}. Подключаем эталонный источник $\Epsilon_N$:
\[\Epsilon_N = (1{,}018 \pm 0{,}005)\ \text{В}\]

Находим примерные значения $R_1$ и $R_2$ по формуле \cref{eq:9} для $\Epsilon = 3\ \text{В}$ и $\Epsilon = 6\ \text{В}$:

\[
R_{1N} = \frac{\Epsilon_N}{\Epsilon}R,\quad R_{1x} = \frac{\Epsilon_x}{\Epsilon}R
\]

Результаты расчетов представлены в \cref{tab:1b}.

\begin{table}[h!]
	\centering
	\caption{Результаты измерений компенсирующих сопротивлений}
	\label{tab:1}
	
	\begin{minipage}{0.45\textwidth}
		\centering
		\subcaption{Экспериментальные данные}
		\label{tab:1a}
		\begin{tabular}{ccc}
			\toprule
			$\Epsilon$, В & $3$ & $6$ \\
			\midrule
			$R_{1N_{\text{ср}}}$, Ом & 3\,781,0 & 1\,880,0 \\
			$R_{1x_{\text{ср}}}$, Ом & 5\,425,0 & 2\,707,5 \\
			\bottomrule
		\end{tabular}
	\end{minipage}
	\hfill
	\begin{minipage}{0.45\textwidth}
		\centering
		\subcaption{Теоретические данные}
		\label{tab:1b}
		\begin{tabular}{ccc}
			\toprule
			$\Epsilon$, В & $3$ & $6$ \\
			\midrule
			$R_{1N}$, Ом & 3\,770 & 1\,885 \\
			$R_{1x}$, Ом & 5\,411 & 2\,706 \\
			\bottomrule
		\end{tabular}
	\end{minipage}
\end{table}

Выставляем ЭДС источника питания $\Epsilon = 3\ \text{В}$, устанавливаем сопротивление $R_3$ на максимальное значение и замыкаем ключ. Методом подбора сопротивлений $R_1$, $R_2$ и постепенного уменьшения $R_3$ находим значение $R_1$, при котором нуль-гальванометр показывает $(+1{,}0 \pm 0{,}5)\ \text{мкА}$, фиксируем $R_1^+$, затем находим $R_1^-$ для $(-1{,}0 \pm 0{,}5)\ \text{мкА}$. Среднее значение вычисляем по формуле:
\[
R_{1N} = \frac{R_1^+ + R_1^-}{2}
\]

Аналогичные измерения проводим для ЭДС источника $\Epsilon = 6\ \text{В}$ и для измеряемой батарейки $\Epsilon_x$. Все полученные данные заносим в таблицы.

\subsection*{Вычисление погрешностей}

Погрешность сопротивлений из характеристик приборов:
\[
R = 11111\ \text{Ом},\quad \Delta R = \pm (0{,}1 + 0{,}2 \frac{m}{R}) \cdot \frac{11111}{100}\ \text{Ом}
\]
где $m$ -- чисто декад магазина, показания которых не равны 0; $R$ -- значение включенного сопротивления в Омах. 
\[
R_{1N_{3\ \text{В}}} = (3781{,}9 \pm 11{,}1)\ \text{Ом},\quad R_{1N_{6\ \text{В}}} = (1888{,}0 \pm 11{,}1)\ \text{Ом}
\]
\[
R_{1x_{3\ \text{В}}} = (5425{,}0 \pm 11{,}1)\ \text{Ом},\quad R_{1x_{6\ \text{В}}} = (2707{,}5 \pm 11{,}1)\ \text{Ом}
\]

Погрешность вольтметра из документации прибора:
\[
\Delta \Epsilon_{xB} = \pm (0{,}15 + 0{,}2 \frac{0{,}2}{1{,}461}) \cdot \frac{1{,}461}{100} \approx 0{,}18\ \text{В}
\]

Погрешность источника питания для $\Epsilon = 3\ \text{В}$ и $\Epsilon_{max} = 99{,}9\ \text{В}$:
\begin{multline*}
	\Delta \Epsilon = \pm (0{,}5\% \cdot \Epsilon + 0{,}1\% \cdot \Epsilon_{max}) =\\= \pm (0{,}005 \cdot 3 + 0{,}001 \cdot 99{,}9)\ \text{В} = \pm 0{,}115\ \text{В}
\end{multline*}

Для $\Epsilon = 6\ \text{В}$:
\[
\Delta \Epsilon = \pm (0{,}005 \cdot 6 + 0{,}001 \cdot 99{,}9)\ \text{В} = \pm 0{,}130\ \text{В}
\]

Из формулы \cref{eq:8} получим:
\[
\Epsilon_x = \frac{\Epsilon R_1 - I_3 \big(r (R_1 + R_2) + R_1 R_2\big)}{R_1 + R_2}
\]

Частные производные:
\[
\frac{\partial \Epsilon_x}{\partial I_3} = - \frac{r (R_1 + R_2) + R_1 R_2}{R_1 + R_2}
\]
\[
\frac{\partial \Epsilon_x}{\partial R_1} = \frac{(\Epsilon - I_3 r - I_3 R_1)(R_1 + R_2) - (\Epsilon R_1 - I_3 (r (R_1 + R_2) + R_1 R_2))}{(R_1 + R_2)^2}
\]
\[
\frac{\partial \Epsilon_x}{\partial R_2} = \frac{R_1(R_1 + R_2 - \Epsilon + I_3 R_2)}{(R_1 + R_2)^2}
\]
\[
\frac{\partial \Epsilon_x}{\partial \Epsilon} = \frac{R_1}{R_1 + R_2}
\]

Погрешность $\Epsilon_x$:
\[
\Delta \Epsilon_x = \left| \frac{\partial \Epsilon_x}{\partial I_3} \right| \Delta I + \left| \frac{\partial \Epsilon_x}{\partial R_1} \right| \Delta R_1 + \left| \frac{\partial \Epsilon_x}{\partial R_2} \right| \Delta R_2 + \left| \frac{\partial \Epsilon_x}{\partial \Epsilon} \right| \Delta \Epsilon
\]
\begin{multline*}
	\Delta \Epsilon_x = \left| - \frac{r(R_1 + R_2) + R_1 R_2}{R_1 + R_2} \right| \Delta I + \left| \frac{R_2 (\Epsilon - I_3 R_1)}{(R_1 + R_2)^2} \right| \Delta R_1 + \\+\left| \frac{R_1(R_1 + R_2 - \Epsilon + I_3 R_2)}{(R_1 + R_2)^2} \right| \Delta R_2 + \left| \frac{R_1}{R_1 + R_2} \right| \Delta \Epsilon
\end{multline*}

Численные значения:
\[
\Delta \Epsilon_{x_1} \approx 0{,}005\ \text{В},\quad \Delta \Epsilon_{x_2} \approx 0{,}003\ \text{В}
\]

Вычислим $\Epsilon_x$ по формуле \cref{eq:10}:

\begin{table}[H]
	\centering
	\caption{Сравнение методов измерения ЭДС}
	\label{tab:2}
	\begin{tabular}{lccc}
		\toprule
		& $\Epsilon$, В & $3$ & $6$ \\
		\midrule
		Метод компенсации & $\Epsilon_{xB}$, В & $1{,}461 \pm 0{,}005$ & $1{,}460 \pm 0{,}003$ \\
		Вольтметр & $\Epsilon_{xB}$, В & $1{,}461 \pm 0{,}18$ & $1{,}461 \pm 0{,}18$ \\
		\bottomrule
	\end{tabular}
\end{table}

Метод компенсации демонстрирует существенно меньшую погрешность по сравнению с прямым измерением вольтметром, что подтверждает его эффективность для точных измерений ЭДС. Полученные результаты хорошо согласуются с теоретическими предсказаниями и свидетельствуют о правильности проведенных экспериментов.